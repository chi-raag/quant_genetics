\PassOptionsToPackage{unicode=true}{hyperref} % options for packages loaded elsewhere
\PassOptionsToPackage{hyphens}{url}
%
\documentclass[]{article}
\usepackage{lmodern}
\usepackage{amssymb,amsmath}
\usepackage{ifxetex,ifluatex}
\usepackage{fixltx2e} % provides \textsubscript
\ifnum 0\ifxetex 1\fi\ifluatex 1\fi=0 % if pdftex
  \usepackage[T1]{fontenc}
  \usepackage[utf8]{inputenc}
  \usepackage{textcomp} % provides euro and other symbols
\else % if luatex or xelatex
  \usepackage{unicode-math}
  \defaultfontfeatures{Ligatures=TeX,Scale=MatchLowercase}
\fi
% use upquote if available, for straight quotes in verbatim environments
\IfFileExists{upquote.sty}{\usepackage{upquote}}{}
% use microtype if available
\IfFileExists{microtype.sty}{%
\usepackage[]{microtype}
\UseMicrotypeSet[protrusion]{basicmath} % disable protrusion for tt fonts
}{}
\IfFileExists{parskip.sty}{%
\usepackage{parskip}
}{% else
\setlength{\parindent}{0pt}
\setlength{\parskip}{6pt plus 2pt minus 1pt}
}
\usepackage{hyperref}
\hypersetup{
            pdftitle={Reflection I},
            pdfauthor={Chiraag Gohel},
            pdfborder={0 0 0},
            breaklinks=true}
\urlstyle{same}  % don't use monospace font for urls
\usepackage[margin=1in]{geometry}
\usepackage{color}
\usepackage{fancyvrb}
\newcommand{\VerbBar}{|}
\newcommand{\VERB}{\Verb[commandchars=\\\{\}]}
\DefineVerbatimEnvironment{Highlighting}{Verbatim}{commandchars=\\\{\}}
% Add ',fontsize=\small' for more characters per line
\usepackage{framed}
\definecolor{shadecolor}{RGB}{248,248,248}
\newenvironment{Shaded}{\begin{snugshade}}{\end{snugshade}}
\newcommand{\AlertTok}[1]{\textcolor[rgb]{0.94,0.16,0.16}{#1}}
\newcommand{\AnnotationTok}[1]{\textcolor[rgb]{0.56,0.35,0.01}{\textbf{\textit{#1}}}}
\newcommand{\AttributeTok}[1]{\textcolor[rgb]{0.77,0.63,0.00}{#1}}
\newcommand{\BaseNTok}[1]{\textcolor[rgb]{0.00,0.00,0.81}{#1}}
\newcommand{\BuiltInTok}[1]{#1}
\newcommand{\CharTok}[1]{\textcolor[rgb]{0.31,0.60,0.02}{#1}}
\newcommand{\CommentTok}[1]{\textcolor[rgb]{0.56,0.35,0.01}{\textit{#1}}}
\newcommand{\CommentVarTok}[1]{\textcolor[rgb]{0.56,0.35,0.01}{\textbf{\textit{#1}}}}
\newcommand{\ConstantTok}[1]{\textcolor[rgb]{0.00,0.00,0.00}{#1}}
\newcommand{\ControlFlowTok}[1]{\textcolor[rgb]{0.13,0.29,0.53}{\textbf{#1}}}
\newcommand{\DataTypeTok}[1]{\textcolor[rgb]{0.13,0.29,0.53}{#1}}
\newcommand{\DecValTok}[1]{\textcolor[rgb]{0.00,0.00,0.81}{#1}}
\newcommand{\DocumentationTok}[1]{\textcolor[rgb]{0.56,0.35,0.01}{\textbf{\textit{#1}}}}
\newcommand{\ErrorTok}[1]{\textcolor[rgb]{0.64,0.00,0.00}{\textbf{#1}}}
\newcommand{\ExtensionTok}[1]{#1}
\newcommand{\FloatTok}[1]{\textcolor[rgb]{0.00,0.00,0.81}{#1}}
\newcommand{\FunctionTok}[1]{\textcolor[rgb]{0.00,0.00,0.00}{#1}}
\newcommand{\ImportTok}[1]{#1}
\newcommand{\InformationTok}[1]{\textcolor[rgb]{0.56,0.35,0.01}{\textbf{\textit{#1}}}}
\newcommand{\KeywordTok}[1]{\textcolor[rgb]{0.13,0.29,0.53}{\textbf{#1}}}
\newcommand{\NormalTok}[1]{#1}
\newcommand{\OperatorTok}[1]{\textcolor[rgb]{0.81,0.36,0.00}{\textbf{#1}}}
\newcommand{\OtherTok}[1]{\textcolor[rgb]{0.56,0.35,0.01}{#1}}
\newcommand{\PreprocessorTok}[1]{\textcolor[rgb]{0.56,0.35,0.01}{\textit{#1}}}
\newcommand{\RegionMarkerTok}[1]{#1}
\newcommand{\SpecialCharTok}[1]{\textcolor[rgb]{0.00,0.00,0.00}{#1}}
\newcommand{\SpecialStringTok}[1]{\textcolor[rgb]{0.31,0.60,0.02}{#1}}
\newcommand{\StringTok}[1]{\textcolor[rgb]{0.31,0.60,0.02}{#1}}
\newcommand{\VariableTok}[1]{\textcolor[rgb]{0.00,0.00,0.00}{#1}}
\newcommand{\VerbatimStringTok}[1]{\textcolor[rgb]{0.31,0.60,0.02}{#1}}
\newcommand{\WarningTok}[1]{\textcolor[rgb]{0.56,0.35,0.01}{\textbf{\textit{#1}}}}
\usepackage{longtable,booktabs}
% Fix footnotes in tables (requires footnote package)
\IfFileExists{footnote.sty}{\usepackage{footnote}\makesavenoteenv{longtable}}{}
\usepackage{graphicx,grffile}
\makeatletter
\def\maxwidth{\ifdim\Gin@nat@width>\linewidth\linewidth\else\Gin@nat@width\fi}
\def\maxheight{\ifdim\Gin@nat@height>\textheight\textheight\else\Gin@nat@height\fi}
\makeatother
% Scale images if necessary, so that they will not overflow the page
% margins by default, and it is still possible to overwrite the defaults
% using explicit options in \includegraphics[width, height, ...]{}
\setkeys{Gin}{width=\maxwidth,height=\maxheight,keepaspectratio}
\setlength{\emergencystretch}{3em}  % prevent overfull lines
\providecommand{\tightlist}{%
  \setlength{\itemsep}{0pt}\setlength{\parskip}{0pt}}
\setcounter{secnumdepth}{0}
% Redefines (sub)paragraphs to behave more like sections
\ifx\paragraph\undefined\else
\let\oldparagraph\paragraph
\renewcommand{\paragraph}[1]{\oldparagraph{#1}\mbox{}}
\fi
\ifx\subparagraph\undefined\else
\let\oldsubparagraph\subparagraph
\renewcommand{\subparagraph}[1]{\oldsubparagraph{#1}\mbox{}}
\fi

% set default figure placement to htbp
\makeatletter
\def\fps@figure{htbp}
\makeatother


\title{Reflection I}
\author{Chiraag Gohel}
\date{2/14/2020}

\begin{document}
\maketitle

\hypertarget{machine-learning-in-genetics-and-genomics}{%
\subsection{Machine Learning in Genetics and
Genomics}\label{machine-learning-in-genetics-and-genomics}}

Both bioinformatics and biostatistics departments have, over the past
decade, invested into the development of machine learning and
statistical methods for the analysis of genomic data. This data varies
in its scope, and could range from population genetics information, to
microbiome data. Algorithms such as PhyloWGS and Canopy seek to model
tumor evolution from multi sampled cell sequencing data. These models
find SNV's between samples, and attempt to recreate a phylogeny. My next
reflection plans to go more in depth into newer biostatistical methods
in genetics and genomics.

\hypertarget{gene-expression-prediction}{%
\subsection{Gene Expression
Prediction}\label{gene-expression-prediction}}

3 years ago, the Tampere University of Technology released a competition
involving the prediction of gene expression from histone modification
signals. Even currently, predicting gene expression from histone
modication signals is a widely studied research topic. The dataset
associated with the dataset is on Primary T CD8+ naive cells from
peripheral blood, or the E047 celltype. Such data was obtained from the
Roadmap Epigenomics Mapping Consortium database.

The dataset consists of training data that consists of:

\begin{itemize}
\tightlist
\item
  A set of data including the analysis of five core histone modification
  marks for a multitude of genes, diving the 10,000 basepair DNA regions
  around the transcription start site into binds of length 100 basepairs
\item
  A set of data with each gene, and whether it demonstrates high
  expression levels, or low expression levels
\end{itemize}

My plan is to use the training data provided to create a model, which
will be used to predict the gene expression levels for the provided test
data set. I will also present the methods used by winning teams for this
competition. Elementary data visualization is presented below:

\hypertarget{summary-statistics}{%
\subsubsection{Summary Statistics}\label{summary-statistics}}

\begin{Shaded}
\begin{Highlighting}[]
\KeywordTok{kable}\NormalTok{(}\KeywordTok{summary}\NormalTok{(x_train))}
\end{Highlighting}
\end{Shaded}

\begin{longtable}[]{@{}lcccccc@{}}
\toprule
& GeneId & H3K4me3 & H3K4me1 & H3K36me3 & H3K9me3 &
H3K27me3\tabularnewline
\midrule
\endhead
& Min. : 1 & Min. : 0.000 & Min. : 0.000 & Min. : 0.000 & Min. : 0.0 &
Min. : 0.000\tabularnewline
& 1st Qu.: 3872 & 1st Qu.: 0.000 & 1st Qu.: 0.000 & 1st Qu.: 0.000 & 1st
Qu.: 0.0 & 1st Qu.: 0.000\tabularnewline
& Median : 7743 & Median : 1.000 & Median : 1.000 & Median : 2.000 &
Median : 1.0 & Median : 1.000\tabularnewline
& Mean : 7743 & Mean : 1.628 & Mean : 1.534 & Mean : 3.368 & Mean : 5.1
& Mean : 1.119\tabularnewline
& 3rd Qu.:11614 & 3rd Qu.: 2.000 & 3rd Qu.: 2.000 & 3rd Qu.: 4.000 & 3rd
Qu.: 3.0 & 3rd Qu.: 2.000\tabularnewline
& Max. :15485 & Max. :163.000 & Max. :93.000 & Max. :106.000 & Max.
:161.0 & Max. :170.000\tabularnewline
\bottomrule
\end{longtable}

\hypertarget{finding-the-average-value-for-h34kme3-for-each-gene-over-sampled-sites-and-visualizing-its-effects-on-gene-expression}{%
\subsubsection{Finding the average value for `H34KME3' for each gene
over sampled sites, and visualizing its effects on gene
expression}\label{finding-the-average-value-for-h34kme3-for-each-gene-over-sampled-sites-and-visualizing-its-effects-on-gene-expression}}

\begin{Shaded}
\begin{Highlighting}[]
\NormalTok{avg_values <-}\StringTok{ }\NormalTok{x_train }\OperatorTok
\StringTok{  }\KeywordTok{group_by}\NormalTok{(GeneId) }\OperatorTok
\StringTok{  }\KeywordTok{summarize}\NormalTok{(}\DataTypeTok{h34kme3 =} \KeywordTok{mean}\NormalTok{(H3K4me3)) }\OperatorTok
\StringTok{  }\KeywordTok{full_join}\NormalTok{(y_train, }\DataTypeTok{by =} \StringTok{"GeneId"}\NormalTok{)}

\KeywordTok{ggplot}\NormalTok{(avg_values, }\KeywordTok{aes}\NormalTok{(h34kme3, Prediction)) }\OperatorTok{+}
\StringTok{  }\KeywordTok{geom_jitter}\NormalTok{(}\DataTypeTok{alpha =} \FloatTok{.4}\NormalTok{) }\OperatorTok{+}
\StringTok{  }\KeywordTok{labs}\NormalTok{(}\DataTypeTok{title =} \StringTok{"Gene Expression by Histone Level"}\NormalTok{)}
\end{Highlighting}
\end{Shaded}

\includegraphics{reflection_i_files/figure-latex/unnamed-chunk-2-1.pdf}

We see that lower gene expression is somewhat correlated with higher
histone values for this specific histone.

\hypertarget{linear-regression}{%
\subsubsection{Linear Regression}\label{linear-regression}}

\begin{Shaded}
\begin{Highlighting}[]
\NormalTok{model}\FloatTok{.1}\NormalTok{ <-}\StringTok{ }\KeywordTok{glm}\NormalTok{(Prediction }\OperatorTok{~}\StringTok{ }\NormalTok{h34kme3, }\DataTypeTok{data =}\NormalTok{ avg_values)}
\KeywordTok{summary}\NormalTok{(model}\FloatTok{.1}\NormalTok{)}
\end{Highlighting}
\end{Shaded}

\begin{verbatim}
## 
## Call:
## glm(formula = Prediction ~ h34kme3, data = avg_values)
## 
## Deviance Residuals: 
##     Min       1Q   Median       3Q      Max  
## -0.7646  -0.4602   0.2824   0.3795   3.3640  
## 
## Coefficients:
##              Estimate Std. Error t value Pr(>|t|)    
## (Intercept)  0.764555   0.005640  135.56   <2e-16 ***
## h34kme3     -0.161904   0.002661  -60.84   <2e-16 ***
## ---
## Signif. codes:  0 '***' 0.001 '**' 0.01 '*' 0.05 '.' 0.1 ' ' 1
## 
## (Dispersion parameter for gaussian family taken to be 0.2017944)
## 
##     Null deviance: 3871.2  on 15484  degrees of freedom
## Residual deviance: 3124.4  on 15483  degrees of freedom
## AIC: 19165
## 
## Number of Fisher Scoring iterations: 2
\end{verbatim}

Our model shows, with significance, that levels of the H34KME3 histone
is negatively correlated with gene expression.

\end{document}
